\subsection{Selection of Architectural Views (Viewpoint)}

Logic view; we’ve chosen a logic view of the overall structure of the game. This is to clarify to the developers how we want to structure the logic. The notation used are as following:\\
\begin{itemize}
	\item Box: used to show the different main states.
	\item Box with lines: substates of the main state.
	\item Rectangles:used to show different options to choose from. 
	\item Arrows: indicates which state you will be taken to next.
\end{itemize}

For the process view we’ve chosen to look closer at the flow of the game when it is played. This view is made with the thought to increase the developers understanding of how the game will be played. The notation is as follows.\\

\begin{itemize}
	\item circle box: used to indicate the start node
	\item  box: used to show the different process.
	\item diamond:used to show different options to choose from. 
	\item arrows: indicates which state you will be taken to next or answers to questions.
\end{itemize}

The develop view shows how the classes are related and where we will be able to make abstractions when more than one class share communalities. Since the only stakeholders are the developers, it’s no big surprise that the development view is made to make it easier for the developers. The notation is as the following. \\

\begin{itemize}
	\item Box: used to indicate classes/groups to make an overview of the class structure and tree
	\item diamond: used to indicate the which classes are in relation to one another.
	\item arrows: used to show which classes belongs to which group.
\end{itemize}
