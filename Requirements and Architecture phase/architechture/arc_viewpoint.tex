\subsection{Selection of Architectural Views (Viewpoint)}

Logic view; we choose a logic view over the overall structure of the game. This is to clarify for the developers how we want to structure the logic. The notation used are as the following:
- Box: used to show the different main states.
- box with lines: substates of the main state.
- Rectangles:used to show different options to choose from. 
- Arrows: indicates which state you will be taken to next.

For the process view we chose to see closer on the flow of the game when it is played. This view is also made with the thought to increases the developers understanding of how the game itself will be played. the notation is as the following.\\

\begin{itemize}
	\item circle box: used to indicate the start node
	\item  box: used to show the different process.
	\item diamond:used to show different options to choose from. 
	\item arrows: indicates which state you will be taken to next or answers to questions.
\end{itemize}


The develop view shows how the classes are related and where we will be able to make abstractions when many classes share communalities. Since the only stakeholders are the developers (us) its no big surprise that the develop view is made to make it easier for the developers. The notation is as the following. \\

\begin{itemize}
	\item Box: used to indicate classes/groups to make an overview of the class structure and tree
	\item diamond: used to indicate the which classes are in relation to one another.
	\item arrows: used to show which classe belongs to which group.
\end{itemize}
